\documentclass[11pt,a4paper]{book}
\usepackage{isabelle,isabellesym}
\usepackage{amssymb}
\usepackage[english]{babel}

% this should be the last package used
\usepackage{pdfsetup}

% urls in roman style, theory text in math-similar italics
\urlstyle{rm}
\isabellestyle{it}


\makeatletter
\newenvironment{abstract}{%
  \small
  \begin{center}%
    {\bfseries \abstractname\vspace{-.5em}\vspace{\z@}}%
  \end{center}%
  \quotation}{\endquotation}
\makeatother

\begin{document}

\title{Simple OCL}
\author{Denis Nikiforov}
\maketitle

\begin{abstract}
  The theory is a simple formalization of the OCL type system,
  its abstract syntax and expressions typing.
  In contrast to Featherweight OCL~\cite{Featherweight_OCL-AFP},
  it's based on a deep embedding approach.

  Simple OCL distincts nullable and non-nullable types. Also
  the theory gives a formal definition of safe navigation
  operations~\cite{DBLP:conf/models/Willink15}.

  The Preliminaries Section of the theory defines a number of
  helper lemmas for transitive closures, tuples and arbitrary
  object models independent from OCL. It allows to use
  Simple OCL as a refenerence theory for formalization of
  analogous languages.
\end{abstract}

\tableofcontents

% include generated text of all theories
\input{session}

\bibliographystyle{abbrv}
\bibliography{root}

\end{document}
